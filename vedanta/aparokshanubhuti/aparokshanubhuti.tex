\documentclass{article}
\usepackage{fontspec}
\usepackage{polyglossia}
\usepackage{framed}
\usepackage{xcolor}
\definecolor{shadecolor}{gray}{0.92}
\setdefaultlanguage{english}
\setotherlanguages{hindi}
\newfontfamily\devanagarifont[Script=Devanagari]{Sanskrit 2003}
\title{Notes on Aparokshânubhuti}
\author{Prithu Goswami}
\date{}

\begin{document}
\maketitle

\begin{shaded}
Aparokshânubhuti is an introductory text about the philosophy of Vedâta written
by Adi Shankaracharya.

These notes are written while listening to Swami Sarvapriyananda's lectures on
Aparokshanubhuti.  In this series of talks, Swami Sarvapriyananda lucidly
unfolds the path to direct "Self-Realization" presented by Adi Shankaracharya
in Aparokshanubhuti.  The text for this series is a translation of Adi
Shankaracharya's Aparokshanubhuti by Swami Vimuktananda. Vidyaranya has written
commentary to this text and Swami Saravpriyananda uses his explanations and
base his explanations off of his for some of the verses.

The Lectures can by found on VedantaNY's Soundcloud:\\ 
{\footnotesize\texttt{https://soundcloud.com/vedantany/sets/aparokshanubhuti-swami-sarvapriyananda}} 
\end{shaded}



\begin{large}
\begin{hindi}
    \begin{center}
   श्रीहरि परमानन्दमुपदेष्टारमीश्र्वरम्  ।\\
  व्यापकं सर्वलोकानां कारणं तं नमाम्यहम् ॥ १ ॥
    \end{center}
\end{hindi}
\end{large}

\texthindi{अहं}
I
\texthindi{परमानन्दं}
Supreme Bliss
\texthindi{उपदेष्टारं}
the First Teacher
\texthindi{ईश्वरं}
Iswara (the Supreme Ruler)
\texthindi{व्यापकं}
All-pervading
\texthindi{सर्वलोकानां}
of all Lokas (worlds)
\texthindi{कारणं}
Cause
\texthindi{तं}
Him
\texthindi{श्रीहरि}
to Sri Hari
\texthindi{नमामि}
bow down.

\bigskip

\textbf{
    1. I$^1$ bow down to Him--to Sri Hari (the destroyer of ignorance), the
    Supreme Bliss, the First Teacher, Iswara, All-pervading One and the
    Cause$^2$ of all Lokas (the universe).
}

{\small
\textit{$^1$I} --- The ego, the Jiva in bondage, who identifies himself with
the gross, subtle and causal bodies, undergoes various sufferings and strives
for liberation.

\textit{$^2$The Cause} --- The efficient as well as the material cause. Just
as a spider weaves its net from the materials of its own body, so does
Iswara create this universe out of Himself.
}

\begin{oframed}
It's a tradition in the vedanta books to praise and bow down to God and
seek his blessings.
\texthindi{श्रीहरि}
is another name of Vishnu. The very nature of Lord is bliss (
\texthindi{परमानन्दं}
- Supreme Bliss).
\texthindi{उपदेष्टारं}
 means the first teacher. All the Gurus have a lineage and all of them point to
God. It is with the grace of God that we learn about spirituality and it's
nature. It starts with God and he is the one who teaches us. They say that a
good Karma leads to one's attainment of knowledge of Brahman.
\texthindi{तं नमाम्यहम्}
- I bow down to you who is all-pervading (
\texthindi{व्यापकं}
) 
 and the Cause of all the worlds (
\texthindi{सर्वलोकानां}
)
\end{oframed}
 
\bigskip

\begin{large}
\begin{center}
    \begin{hindi}
	अपरोक्षानुभूतिर्वै प्रोच्यते मोक्षसिध्दये ।\\
	सद्भिरेव प्रयत्नेन वीक्षणीया मुहुर्मुहुः ॥ २ ॥
    \end{hindi}
\end{center}
\end{large}

\texthindi{मोक्षसिद्धये}
For the acquisition of final liberation (from the bondage of ignorance)
\texthindi{वै}
(expletive)
\texthindi{अपरोक्षानुभूतिः}
(the means of attaining to) Self-realization
(\texthindi{अस्माभिः}
by us)
\texthindi{प्रोच्यते}
is spoken of in detail
\texthindi{सद्भिः}
by the pure in heart
\texthindi{एव}
only
(\texthindi{इयं}
this)
\texthindi{प्रयत्नेन}
with all effort
\texthindi{मुहुर्मुहुः}
again and again
\texthindi{वीक्षणीया}
should be meditated upon.

\bigskip

\textbf{ 2. Herein is expounded (the means of attaining to) Aparokshânubhuti
$^1$(Self-realization) for the acquisition of final liberation.  Only the pure
in heart should constantly and with all effort meditate upon the truth herein
taught.  }

{\small \textit{$^1$ Aparokshânubhuti} --- It is the direct cognition of the
Âtman which is always present in all thought.

Everybody has some knowledge of this Âtman or Self, for, to deny the Self is to
deny one's own existence. But at first its real nature is not known. Later on,
when the mind becomes purer through Upâsanâ and Tapas, the veil of ignorance is
gradually withdrawn and the Self begins to reveal its real nature. A higher
knowledge follows at an advanced stage, when the knowledge of the `Self as mere
witness' is seen as absorbing all other thoughts.

But the end is not yet reached. The idea of duality, such as `I am the witness'
(`I' and the `witness'), is still persisting. It is only at the last stage when
the knower and the known merge in the Self-effulgent Âtman, which alone ever
\textit{is}, and besides which nothing else exists, that the culmination is
reached. This realization of the \textit{non-dual} is the consummation of
Aparokshânubhuti.

It is needles to say that Aparokshânubhuti may here mean also the work that
deals with it.  }

\begin{oframed}
For attaining self-realization, one must intently enquire about what is being
taught here (The Aparokshânubhuti). There are three categories of experience
\texthindi{(अनुभूतिः)}
\texthindi{- प्रत्यक्षः, परोक्षः, अपरोक्षः -}
\texthindi{प्रत्यक्षः}
means sense-perception. Everything we hear, everything we see, taste, etc
through our sense organs is
\texthindi{प्रत्यक्षः}.
\texthindi{अक्षः} literaly means the eyes but also means
the sense organs. The knowledge and experience we gain
through our sense organs is
\texthindi{प्रत्यक्षः}.
There are five of these senses.  We also get knowledge in another way, we think
about the things we experience,  we infere throught observations we have . The
whole of science is this. Observation and inference. Perceptions can be made
also using other tools that the humans have inventeded. So this knoweledge we
get based on what we see/hear after drawing conclusions is called
\texthindi{परोक्षः}
. Like atoms for example. We don't see them but we have drawn a conclusion that
everything is made up of atoms through experiments and different theories
by many people. All of religion is also Paroksha. There it's the beliefs that
are present in the religions that we develop on the basis of faith. Paroksha
is something 'beyond senses'. But Vedânta is neither Paroksha or Pratyksha.
\texthindi{अपरोक्षः}
is something beyond Prathksha and Paroksha. Its the very pure consciousness
itself. It's an attribute of Brahman.
The basis of our senses is because of 
\texthindi{अपरोक्षः}
. And the self is pure consiousness and bliss which is
\texthindi{अपरोक्षः}
It's all here. This is Brahman but we need to realise it and that is what
Aparokshânubhuti is. Teaching in Vedâta is all about pointing out. It points
out what is already there - Brahman.
Those who are pure in heart should constantly meditate with all effort again
and again.

\end{oframed}


\begin{large}
\begin{center}
    \begin{hindi}
	स्ववर्णाश्रमधर्मेण तपसा हरितोषणात् ।\\
	साधनं प्रभवेत् पुंसां वैराग्यादिचतुष्टयम् ॥ ३ ॥
    \end{hindi}
\end{center}
\end{large}

\texthindi{स्ववर्णाश्रमधर्मेण}
By the performance of duties pertaining to one's social order and stage in
life
\texthindi{तपसा}
by austerity
\texthindi{हरितोषणात्}
by propitiating Hari (the Lord)
\texthindi{पुंसां}
of men
\texthindi{वैराग्यादि}
Vairâgya (dispassion) and the like
\texthindi{चतुष्टयं}
the four-fold
\texthindi{साधनं}
means (to knowledge)
\texthindi{प्रभवेत्}
arises.

\bigskip

\textbf{ 3. The four preliminary qualifications$^1$ (the means to the
attainment of knowledge), such as Vairâgya (dispassion) and the like, are
acquired by men by propitiating Hari (the Lord), through austerities and the
performance of duties pertaining to their social order and stage in life.  }

{\small
\textit{$^1$The four preliminary qualifications} --- There are
\texthindi{वैराग्यं}
dispassion,
\texthindi{विवेकः}
discrimination,
\texthindi{शमादिषट्सम्पत्तिः}
six treasures such as Sama (the control of the mind) and the like, and
\texthindi{मुमुक्षुत्वं}
yearning for liberation (from the bondage of ignorance).
}

\begin{oframed}

There are some duties and responsibilities of a person in their social life.
One must not give up his life to persue the quest of knowing oneself. There is
a social obgligation and it's good not give them up. One must complete what one
has started. There were stages in life where there were certain duties and
responsibilies of a man. Shankaracharya says that the Vernacular system that
was setup in ancient India was for the very purpose of realizing Brahman. And
they do help in one or other way in realizing non-dualism. One must not disturb
the external world and lead a disciplined life and cultivate realization at the
same time.

There are four preliminary qualification required -
\texthindi{वैराग्यं, विवेकः, शमादिषट्सम्पत्तिः, मुमुक्षुत्वं}

\texthindi{विवेकः}
- Differentiation between the eternal and temporary.

\texthindi{वैराग्यं}
- Dispassion for the temporary and a desire to know the eternal.

\texthindi{शमादिषट्सम्पत्ति}
- The six treasures/disciplines.

\texthindi{मुमुक्षुत्वं}
- A strong desire for spirituality and liberation.

\end{oframed}

\begin{large}
\begin{center}
    \begin{hindi}
	ब्रह्मादिस्थावरान्तेषु वैराग्यं विषयेष्वनु ।\\
	यथैव काकविष्ठायां वैराग्यं तद्धि निर्मलम् ॥ ४ ॥
    \end{hindi}
\end{center}
\end{large}

\textbf{4. The indifference with which one treats the excreta of a crow - such an
indifference to all objects of enjoyment from the realm of Brahmâ to this
world (in view of their perishable nature), is verily called pure Variâgya.
}

\smallskip

\begin{oframed}
As we want to be happy, we see avenues of happiness in this world. We come to a
realization that none of them provide permanent satisfaction. There are
satisfactions in life but they are not the purpose in life. One who is
spiritually inclined should not persue these pleasures in world. Also the
concept of heaven should be dismissed as it too is temprary pleasure and
spiritulaity is beyond that too.
\end{oframed}




\begin{large}
\begin{center}
    \begin{hindi}
	नित्यमात्मस्वरुपं हि द्रश्यं तद्विपरितगम् ।\\
	एवं यो निश्र्चयः सम्यग्विवेको वस्तुनः स वै ॥ ५ ॥
    \end{hindi}
\end{center}
\end{large}

\textbf{5. Âtman (the seer) in itself is alone permanent, the seen is opposed
to it (i.e. transient) - such a settled conviction is truly known as discrimination (Viveka)
}

\begin{oframed}
\texthindi{विवेकः}
means `to separate'.  Life serves both, the eternal and the termperal.
The ability to discriminate in life between the two is
\texthindi{विवेकः}
There is a eternal truth and having the clarity is what is important. Feeling
that there is something to this spiritual life is 
\texthindi{विवेकः}.
\end{oframed}


\begin{large}
\begin{center}
    \begin{hindi}
	सदैव वासनात्यागः शमोऽयमिति शब्दितः ।\\
	निग्रहो बाह्यवृत्तीनां दम इत्यभिधीयते ॥ ६ ॥
    \end{hindi}
\end{center}
\end{large}

\textbf{6. Abandonment of desires at all times is called Sama and restraint
of the external functions of the organs is called Dama.
}

\begin{oframed}
Quitening of mind is what is Sama (or control of mind) and Dama is the control
of the external body or the motor organs. Quitening of mind is not possible
with the mind filled with various desires.
6 treasure: 
1. Sama
2. Dama
3. Uparati
4. Titiksha
5. Sradha
6. Samadhana
\end{oframed}

\begin{large}
\begin{center}
    \begin{hindi}
	विषयेभ्यः परावृत्तिः परमोपरतिर्हि सा ।\\
	सहनं सर्वदुःखानां तितिक्षा सा शुमा मता ॥ ७ ॥
    \end{hindi}
\end{center}
\end{large}

\textbf{
7. Turning away completely from all sense-objects is the height of Uparati,
and patient endurance of all sorrow or pain is known as Titikshâ which is
conducive to happiness.
}

\begin{oframed}
Uparati is the opposite of \textsl{rati} - the enjoyment of worldly things
through the five senses. Not seeking pleasures outside but within is Uparati.
One must know to be to himself in solitude. Endurance is also important
(Titikshâ) as the world will try to draw you away from spirituality.
\end{oframed}



\begin{large}
\begin{center}
    \begin{hindi}
	निगमाचार्यवाक्येषु भक्तिःश्रद्धेति विश्रुता ।\\
	चित्तैकाग्रयं तु सल्लक्ष्ये समाधानमिति स्मृतम् ॥ ८ ॥
    \end{hindi}
\end{center}
\end{large}

\textbf{8. Implicit faith in the words of the Vedas and the teachers (
who interpret them) is known as Sraddhâ, and concentration of the mind on the
only object Sat (i.e. Brahman) is regarded as Samâdhâna.
}

\begin{large}
\begin{center}
    \begin{hindi}
	संसाराबंधनिर्मुक्तिः कथं मे स्यात् कदा विधे ।\\
	इति या सु्दृढा बुद्धिर्वक्तव्या सा मुमुक्षुता ॥ ९ ॥
    \end{hindi}
\end{center}
\end{large}

\textbf{9. When and how shall I, O Lord, be free from the bonds of this world
(i.e. births and deaths) - such a burning desire is called Mumukshutâ.
}

\bigskip

\begin{large}
\begin{center}
    \begin{hindi}
	उक्तसाघनयुक्तेन विचारः पुरुषेण हि।\\
	कर्तव्यो ज्ञानसिद्धयर्थमात्मनः शुभमिच्छता ॥ १० ॥
    \end{hindi}
\end{center}
\end{large}

\textbf{ 10. Only that person who is in possession of the said qualifications
(as means to Knowledge) should constantly reflect$^1$ with a view to attaining
Knowledge, desiring his own good$^2$.
}

{\small\textit{$^1$ Should constantly reflect} --- After a person has
attained the tranquillity of the mind through Sâdhanaâs, he should strive hard
to maintain the same by constantly reflecting on the evanescent nature of this
world and withal dwelling on the highest Truth till he becomes one with It.}

{\small\textit{$^2$ Good} --- The highest good, i.e liberation from the bondage
of ignorance.}

\bigskip

\begin{large}
\begin{center}
    \begin{hindi}

	नोत्पधते विना ज्ञानं  विचारेणान्यसाधनैः।\\
    यथा पदार्थभानं हि प्रकाशेन क्वचित् ॥ ११  ॥

    \end{hindi}
\end{center}
\end{large}

\textbf{11. Knowledge is not brought about by any other means$^1$ than Vichâra,
just as an object is nowhere perceived (seen) without the help of light.}

{\small\textit{$^1$ By any means} --- By Karma, Upâsanâ and the like. It is ignorance or Avidyâ which has witheld the light of Knowledge from us. To get at Knowledge, therefore, we have to remove this Avidyâ. But so long as we are engaged when we make an enquiry into the real nature of this Avidyâ that it gradually withdraws and at last vanishes; then alone Knowledge shies.}
\begin{oframed}
Knowing yourself will lead you to happines. We are sad because we don't know
who we are (Avidyâ).  Happines is an internal state and someone can be happy
even if he/she is suffereing. And someone who has everything externally can
also be quite sad. And there are examples of this in the world.  Ignorance
about the self leads to actions in the world as we are looking for happines.
Action leads to consequence and we are trapped in the cycle of birth and death.
Ignorace > Desire > Action. Once there is action we are trapped in 
\texthindi{संसारा}
. Knowledge removes the ignorance, and you realize yourself as being complete
and all the desires to attain happines from the world are no longer needed.
\end{oframed}



\bigskip

\begin{large}
\begin{center}
    \begin{hindi}

	कोऽहं कथमिदं जातं को वै कर्ताऽस्य विधते।\\
    उपादानं किमस्तीह विचारः सोऽयमीदृशः ॥ १२ ॥

    \end{hindi}
\end{center}
\end{large}

\textbf{ 12. Who am I?$^1$ How is this (world) created? Who is its creator? Of
what material is this (world) made? This is the way of that Vichâra$^2$
(enquiry) }

{\small\textit{$^1$Who am I?} --- We know that we are, but we do not
know what our real \textit{nature} is. In the waking state we think that we are
the body, the physical being, and consquently feel ourselves strong or weak,
young or old. At another time, in the dream state, regardless of the physical
existence we remain only in a mental state, where we are merely thinking beings
and feel only the misery or happiness that our thoughts create for us. Again,
in deep sleep, we enter into a state where we cannot find the least trace of
any such attribute whereby we can either assert or deny our existence.

We pass through these states almost daily and yet do not know which of them
conforms to our real nature. So the question, 'Who am I?' is always with us an
unsolved riddle. It is, therefore, necessary to investigate into it.

\textit{$^2$This is the way of that Vichâra} --- It is said in the preceding
Sloka that Knowledge is attainable by no other means but Vichâra or an enquiry
into the Truth. Herein is inculcated in detail the method of such an enquiry }


\bigskip


\begin{oframed}
This Sloka sets forth the main theme of the text.
There are enquiry of three entities:
\begin{enumerate}
    \item{
            \texthindi{जीवः}
            - Who am I?
        }
    \item{
            \texthindi{जगतः}
            - What is this world?
        }
    \item{
            \texthindi{ईश्वरं}
            - Who is God/Brhaman? What is his/its nature?
            This question is divided into two: 1. The material cause and
            2. Efficient cause (the intelligence who made the universe)

        }
\end{enumerate}


Taking the example of the clay pot - Is the clay in the pot? Well no. The whole
pot \textit{is} the clay. The clay is in the \textit{form} of the pot.  And
there exists no such thing as a pot apart from the clay. Just like that, we
are the real nature (Brahman), we don't exist apart from it. It is our
reality. Also, is there a pot in the clay? No. The clay is the only thing
there is and it, as well, is made up of atoms and molecules. But we can't say
the pot is in the clay. The clay has no knowledge of the pot. It exists
without the pot, but the pot cannot exist without the clay. The same way -
Is the universe in consciousness? or the consciousnes in the universe? The
pot is a name and form and is imagined in the clay. In the same way the
fundamentalss of the universe - time, space, causation are imagined in pure
consciousness (Brahman). You'll know that there is no pot in the clay,
there is only clay. We will see that all of this universe, including us is
something that is imagined/superimposed/projected in consciousness.
Consciousness alone is real.

From the next verse, the three enquiries are taken up and go on till the end of
    the text.

\end{oframed}

\bigskip

\begin{large}
\begin{center}
    \begin{hindi}

    नाहं भूतगणो देहो नाहं चाक्षगणस्तथा ।\\
    एतद्विलक्षणः कश्चिद्विचार सोऽयमीदृशः॥ १३ ॥

    \end{hindi}
\end{center}
\end{large}

\textbf{ 13. I am neither the body$^1$, a combination of the (five) elements
(of matter), nore am I an aggregate of the senses; I am something different
from these. This is the way of that Vichâra.  }

{\small\textit{$^1$I am neither the body} --- This body has its origin in
insentient matter and as such it is devoid of consciousness. If I be the body,
I should be unconscious; but by no means am I so. Therefore I cannot be the
body.}

\begin{oframed}

We think we are the body-mind system. Vedanta assures that we are mistaken by
this.  This body is not ours and when we claim that the body is ours there
is suffering. The body is made of the five elements(atoms, subatomic
particles according to  today's science) which we did not make and we don't
own them thus the body is also not ours. We don't own the materials and so
we can't claim the body to be ours or say that `I am this body'.

[Swam Sarvapriyananda uses Vidyarnya's commentaries here] - The experiencer
and the experienced are two different things and I being the experiencer,
experience things in this world. I am the experiencer of my physical body
and hence I cannot be experienced object i.e. The body.  The materialists
say `If you are not the physical body, why don't we say that you are the
mind and the sense organs as you use them to experience your body'. But the
sense organs themselves are the objects of our experience.  The mind can
also be the object of our experiences just like any other objec in the
world. Therefore I am not even the sense organs and the mind.  The mind is
also an object, an instrument. As I am the experiencer and I experience
something, I cannot be the experienced object.  The mind and the sense
organs are just instruments to experience the world and infact themselves
as well. I am something other than these objects (mind and sense organs). I
am very different from all of this, the body and the mind. `something'
becuase pure consciousness cannot be expressed with the use of language.

\end{oframed}


\bigskip

\begin{large}
\begin{center}
    \begin{hindi}
    अज्ञानप्रभवं सर्व ज्ञानेन प्रविलीयते ।\\
    संकल्पो विविधः कर्ता विचारः सोऽयमीदृशः ॥ १४ ॥
    \end{hindi}
\end{center}
\end{large}

\texthindi{सवं }
Everything
\texthindi{अज्ञानप्रभवं}
produced by ignorance
(\texthindi{अस्ति}
is)
\texthindi{ज्ञानेन}
through Knowledge
(\texthindi{तत् }
that)
\texthindi{ प्रविलीयते }
completely disappears
\texthindi{विविधः }
various 
\texthindi{ संकल्पः}
thought
\texthindi{ कर्ता}
creator
(\texthindi{भवति }
is )
\texthindi{सोऽयं, }
etc.

\bigskip

\textbf{14. Everything is produced by ignorance,$^1$ and dissolves in the wake
of Knowledge. The various thoughts (modifications of Antahkarana) must be the
creator.$^2$ Such is this Vichâra.
}


{\small
\textit{$^1$Everything is produced by ignorance} --- In reply to the
question in Sloka 12 as to the caus of this world it is here said that
ignorance is the cause of everything.

Sometimes seeing something coiled up on the road we mistake it for a snake and
shrink back out of fear. But afterwards when we discover that it is nothing but
a piece of rope, the question arises in the mind as to the cause of the
appearance of the snake. On enquiry we find that the cause of it lies nowhere
else than in our ignorance of the true nature of the rope. So also the cause of
the phenomenal world that we see before us lies in the ignorance or Mâyâ that
covers the reality.

\textit{$^2$The various thoughts ... the creator} --- The only thing that we
are directly aware of is our own thoughts. The world that we see before us is
what our thoughts have created for us. This is clearly understood when we
analyse our experiences in dreams. There the so-called material world is
altogether absent, and yet the thoughts alone create a world which is as
material as the world now before us. It is, therefore, held that the whole
universe is, in the same way, but a creation of our thoughts.
}


\begin{oframed}

This Sloka is answering the question `What is this world?'. The reason we
experience this world is due to ignorance, just like we don't know the
reality of a rope lying and mistake it to be snake.  Whatever we know
through ignorance is false, just like how we see a mirage in the dessert or
the snake in the rope. The sky seems blue but it's due to scattering of the
light and it appears to be blue. It not actually is blue. That whcih is
falsed doesn't mean it can't be experience, it's just that it doesn't exist
but still it appears. The water isn't in the dessert but it appears, the
snake isn't there but it appears as the rope. Just like the world appears
but doesn't exist, although we experience it. Even if it's false we still
experience but we realize the truth. Just like that we realize Brahaman but
we still continue to experience the world.  What we see as the world is in
reallity Brahman. Upon enlightenment we realize this.

Which Knowledge removes which Ignorance? Knowledge will remove ignorance only
if the locus and object of the knowledge and ignorance is the same. For
example - If I don't know Physics I have Ignorance in my mind(locus) about
Physics (object). So there must be a Knowledge in my mind(locus) about
Physics (object) to remove the ignorance. When there is Knowledge, the
ignorance disappears in a flash just like how we strike a match and
darkness of a thousand years disappears in a flash.

\end{oframed}

\bigskip

\begin{large}
\begin{center}
    \begin{hindi}
    एतयोर्यदुपादानमेकं सूक्षमं सदव्ययम् ।\\
    यथैव मृद्घटादीनां विचारः सोऽयमीदृशः ॥ १५ ॥
    \end{hindi}
\end{center}
\end{large}

\texthindi{यथैव }
Just as
\texthindi{घटादीनां }
{of the pot and the like }
(\texthindi{उपादानं }
material)
\texthindi{मृत् }
earth
(\texthindi{भवति }
is, 
\texthindi{तथैव }
so also)
\texthindi{एतयोः }
{of these two }
\texthindi{यत् }
{which }
\texthindi{उपादानं }
{material }
(\texthindi{तत् }
that)
\texthindi{एकं }
one
\texthindi{सक्षमं }
{subtle }
\texthindi{अव्ययं }
{unchanging }
\texthindi{सत् }
Sat (Existence)
(\texthindi{अस्ति}
is)
\texthindi{सोऽयं , }
etc.

\bigskip

\textbf{
    15. The material (cause) of these two (i.e ignorance and thought) is the one$^1$ (without a second), subtle (not apprehended by the senses) and unchanging Sat (Existence), just as the earth is the material (cause) of the pot and the like. This is they way of that Vichâra.
}

{\small \textit{$^1$ One} --- Because it does not admit of a second of the same
or of a different kind, or of any parts within itself. It is one homogeneous
whole.  }

\begin{oframed}
    We have Karma and we need the world to experience and pay for the Karma
    because of Desires. Thinking
    \texthindi{(संकल्पः)}
    leads to Desires
    \texthindi{(कामः)}
    and Desires leads to Karma
    \texthindi{(कर्मः)}.
    God gives us this world in order to give us opportunities to experience
    the results of our Karma.

    What is the material cause of this universe? It is Sat (Existence).  We are
    in an ocean of Existence. And Existence in itself is Brahman. The material
    cause of something is that which gives it existence. The clay is the
    material cause of the pot as it gives it existence. Just like that, the
    material cause of the entire universe is Sat (Existence) or Brahman. Also,
    both the material cause and efficent cause of the universe is Brahman.
    Brahman itself made the universe happen (Check footnote \#2 of the first
    verse). Brahman alone is real, the universe is an appearance, and you, the
    individual, are Brahman.

\end{oframed}


\begin{large}
\begin{center}
    \begin{hindi}
    अहमेकोऽपि सूक्ष्मश्च ज्ञाता साक्षी सदव्ययः ।\\
    तदहं नात्र सन्देहो विचारः सोऽयमीदृशः ॥ १६ ॥
    \end{hindi}
\end{center}
\end{large}

(\texthindi{यस्मात्}
Because)
\texthindi{अहं}
I
\texthindi{अपि}
also
\texthindi{एकः}
one
\texthindi{सूक्ष्मः}
the subtle
\texthindi{च}
(expletive)
\texthindi{ज्ञाता}
the Knower
\texthindi{साक्षी}
the Witness
\texthindi{सत्}
the Existent
\texthindi{अव्ययः}
the Unchanging
(\texthindi{अस्मि}
am,
\texthindi{तस्मात्}
therefore)
\texthindi{अहं}
I
\texthindi{तत्}
``That"
(\texthindi{अस्मी}
am)
\texthindi{अत्र}
here
\texthindi{सन्देहः}
doubt
\texthindi{न}
not
(\texthindi{अस्ति}
is)
\texthindi{सोऽयं,}
etc.

\bigskip

\textbf{ 16. As I am also the One, the Subtle, the Knower,$^1$ the Witness, the
Ever-Existent and the Unchanging, so there is no doubt that I am ``That"$^2$
(i.e. Brahman). Such is this enquiry.  }

{\small \textit{$^1$The Knower} --- The supreme Knower who is ever present in
all our perceptions as consciousness, and who perceieves even the ego.

When I say, ``I know that I exist," the ``I" of the clause `that I exist' forms
a part of the predicate and as such it cannot be the same `I' which is the
subject. This predicative `I' is the ego, the object. The subjective `I' is the
supreme Knower.

\textit{$^2$ I am ``That"} --- I, the ego, when stripped of all its limiting
adjuncts, such as the body and the like, becomes one with ``That," the supreme
Ego, i.e. Brahman. In fact, it is always Brahman ; Its limitation being but the
creation of ignorance.  }

\begin{oframed}

    In advaita vedanta there are three things to remember: 1. Don't identify
    yourself with anything that is an object and that is limited. 2.  Know
    yourself as Brahman. See the subject as the infinite Brahman. 3. Look upon
    the universe and recognize everything in the universe as you yourself, as
    everything is Brahaman.  The unchaning consiousness is the Witness, is also
    the real us. The Knower is also us but limited to the body.The knower uses
    the mind and the sense organs. The Witness on the other hand, doesn't
    require these instruments of the body.

\end{oframed}

\begin{large}
\begin{center}
    \begin{hindi}

आत्मा विनिष्कलो ह्येको देहो बहुभिरावृतः ।\\
तयोरैक्यं प्रपश्यन्ति किमज्ञानमतः परम् ॥ १७ ॥
    \end{hindi}
\end{center}
\end{large}

\texthindi{आत्मा}
Âtman
\texthindi{हि}
verily
\texthindi{एकः}
one
\texthindi{विनिष्कलः}
without parts
(\texthindi{अस्ति}
is)
\texthindi{देहः}
the body
\texthindi{बहुभिः}
by many (parts)
\texthindi{आवृतः}
covered
(\texthindi{भवति}
is,
\texthindi{मृढाः}
the ignorant)
\texthindi{तयॊः}
of these two
\texthindi{ऐक्यं}
identity
\texthindi{प्रपश्यन्ति}
see (confound)
\texthindi{अतःपरम्}
else than this
\texthindi{किम्}
what
\texthindi{अज्ञानं}
ignorance
(\texthindi{अस्ति}
is)

\bigskip

\textbf{ 17. Âtman is verily one and without parts, whereas the body consists
of many parts; and yet people see (confound) these two as one! What else can be
called ignorance but this?$^1$ }

{\small \textit{$^1$What else can be called
ignorance but this?} --- To give rise to confusion in knowledge is a unique
characteristic of ignorance. It is through the influence of ignoracne that one
confounds a rope with a snake, a mother-of-pearl with a piece of silver and so
on. But, after all, the power of ignorance is not completely manifest there;
for one could easily find an excuse for such confusions when there exist some
common characteristics between the real and the apparant. The nature of
ignorance is, however, fully revealed when one confounds the subject (i.e.
Âtman) with the object (i.e. the body), which have nothing in common between
them, being opposed to each other in all respects.  }


\begin{oframed}

In the first stage we are trying to find the true nature of `I'. Our
    understanind of the self must be refined. From here on we are going to do
    just that - refine the meaning of `I'. Subject and Object (I and this).
    Whatever we say is this is not I. The problem start with the this (pointing
    to body). We include the body in the meaning of `I'. The body is an object
    of experience. You can feel it. You can hear it. You can smell it.
    Everything that has `This'(object) has to be excluded from the
    understanding of `I'.

    Identification with the body is a problem. How to loose that
    identification? Swamiji gives three things to contemplate. We consider
    ourself as `one' but we are composed of so many parts - the hand, the legs,
    the eyes, and all the other various organs, tissues, cells, etc - but we
    stil say or feel that `I am one'. You need to contemplate this distinction,
    that even after consisting of so many things you feel that you are
    `one'.Second, I alsways feel that I sensient and the body is the object of
    my experience. Third, I always feel that I am the same and the body is
    changing.

\end{oframed}


\begin{large}
\begin{center}
    \begin{hindi}
    आत्मा नियामकश्चान्तर्देहो बाह्यो नियम्यकः ।\\
    तयोरैक्यं प्रपश्यन्ति किमज्ञानमतः परम् ॥ १८ ॥
    \end{hindi}
\end{center}
\end{large}

\texthindi{आत्मा}
Âtman
\texthindi{नियामकः}
the ruler
\texthindi{अन्तः}
internal
\texthindi{च}
and
(\texthindi{भवति}
is)
\texthindi{देहः}
the body
\texthindi{नियम्यकः}
the ruled
\texthindi{वाह्यः}
external
(\texthindi{भवति}
is)
\texthindi{तयोरैक्यं,}
etc.

\bigskip

\textbf{18.  Âtman is the ruler of the body and internal, the body is the ruled
and external; and yet, etc. }

\begin{oframed}

    Shankaracharya is pointing out here with his arguments. The body is
    something that is controlled and we are the controller of that body. 
    The atman `lends' consciouness to the body and mind.

    The body is something that is outside and I am something inside.  The
    feeling of `I' we have is also infact an object and therefore it is not us.

\end{oframed}

\begin{large}
\begin{center}
    \begin{hindi}
आत्मा ज्ञानमयः पुण्यो देहो मांसमयोऽशुचिः ।\\
तयोरैक्यं प्रपश्यन्ति किमज्ञानमतः परम् ॥ १९ ॥
    \end{hindi}
\end{center}
\end{large}

\texthindi{आत्मा}
Âtman
\texthindi{ज्ञानमयः}
all consciousness
\texthindi{पुण्यः}
holy
(\texthindi{भवति}
is)
\texthindi{देहः}
the body
\texthindi{मांसमयः}
all flesh
\texthindi{अशुचिः}
impure
(\texthindi{भवति}
is)
\texthindi{तयोरैक्यं,}
etc.

\bigskip

\textbf{
    19. Âtman is all consiousness and holy, the body is all flesh and impure;
    and yet, etc.
}


\begin{large}
\begin{center}
    \begin{hindi}
आत्मा प्रकाशकः स्वच्छो देहस्तामस उच्यते ।\\
तयोरैक्यं प्रपश्यन्ति किमज्ञानमतः परम् ॥ २० ॥
    \end{hindi}
\end{center}
\end{large}

\texthindi{आत्मा}
Âtman
\texthindi{प्रकाशकः}
the Illuminator
\texthindi{स्वच्छः}
pure
\texthindi{देहः}
the body
\texthindi{तामसः}
of the nature of darkness
\texthindi{उच्यते}
is said
\texthindi{तयोरैक्यं,}
etc.

\bigskip

\textbf{20. Âtman is the (supreme) Illuminator and purity itself; the body is
said to be of the nature of darkness; and yet, etc.}

\begin{oframed}

    Just like how the light that shines upon an object and is not affected by
    it, in the same way consiousness is not affected by what it illumines. It
    can reveal a bad thing or a good thing and is not affected by it.
    Consiousness is the lights of lights and because of it everything is
    revealed. Everything we know about is because of consiousness and therefore
    consiousness transcends all that we know. Consiousness is just the
    illuminer and is apart from all that illumines. Now saying that "this"
    body is "me" is the most ridiculous thing.

\end{oframed}

\begin{large}
\begin{center}
    \begin{hindi}
    आत्मा नित्यो हि सद्रूपो देहोऽनित्यो ह्यसन्मयः ।\\
    तयोरैक्यं प्रपश्यन्ति किमज्ञानमतः परम् ॥ २१ ॥
    \end{hindi}
\end{center}
\end{large}

\texthindi{आत्मा}
Âtman
\texthindi{नित्यः}
eternal
\texthindi{हि}
since
\texthindi{सद्रूपः}
Existence itself
\texthindi{देहः}
the body
\texthindi{अनित्यः}
transient
\texthindi{हि}
because
\texthindi{असन्मयः}
non-existence in essence
\texthindi{तयोरैक्यं,}
etc

\bigskip

\textbf{21.Âtman is eternal, since it is Existence itself: the body is
transient, as it is non-existence in essence;$^1$ and yet, etc.
}

{\small \textit{$^1$The body is ... non-existence in essence} --- The
body is undergoing change at every moment, and as such, cannot be eternal.
But granting that it is non-eternal, how can it be non-existent? -- for, so
long as it lasts we surely see it as existing.

At first sight the body appears to be existing, however temporary its existence
may by. A relative existence (Vyavahârika Sattâ) is, therefore ascribed to it.
But when one examines it and tries to find out its real nature, this so-called
tangible body gradually becomes attenuated and at last disappears altogether.
It is, therefore, said here that the body, as such, is always non-existent,
even though it may appear as existing for a time to those who do not care to
see it through.

}


\begin{oframed}

    Any change to be experienced, it should be caused to something that can be
    experienced by the experiencer. So to feel the change, the change must
    happen to the object of the experiencer and death is such a change. The
    experiencer hence is not affected by any change.  Swamiji takes an example
    of a potato being boiled. When it gets hot, we say that it's a hot potato.
    It borrows heat from the boiling water and the water inturn borrows heat
    from the fire. But neither the potato nor the water claim heat as their
    intrinsic property. That heat soley belongs to the fire. The potato and the
    water just borrow the heat from the fire. in the same way, existence is
    borrowed and we are born with this body. Just like how the potato and the
    water will eventually loose their borrowed heat, in the same way this body
    will stop existising as it had borrowed its existence from Consiousness.
    Something that has existence as its intrinsic property will forever be
    immortal. Such a thing is called Sat.

\end{oframed}

\begin{large}
\begin{center}
    \begin{hindi}
    आत्मनस्तत्प्रकाशत्वं यत्पदार्थावभासनम् ।\\
    नाज्ञादिदीप्तिवद्दीप्तिर्भवत्यान्ध्यं यतो निशि ॥ २२ ॥
    \end{hindi}
\end{center}
\end{large}

\texthindi{यत्}
Which
\texthindi{पदार्थावभासनं}
manifestation of all objects
\texthindi{तत्}
that
\texthindi{आत्मनः}
of Âtman
\texthindi{प्रकाशत्वं}
illumination
\texthindi{न}
not
\texthindi{अभयादिदीप्तिवत्}
like the light of fire and the rest (
\texthindi{आत्मनः}
of Âtman )
\texthindi{दीप्तिः}
light (
\texthindi{भवति}
is )
\texthindi{यतः}
for
\texthindi{निशि}
at night
\texthindi{आन्ध्यं}
darkness
\texthindi{भवति}
exists.

\bigskip

\textbf{22. The luminosity of Âtman consits in the manifestation of all
objects. Its luminosity is not$^1$ like that of fire or any such thing, for (in
spite of the presence of such lights) darkness prevails at night (at some place
or other).}

{\small \textit{$^1$Its luminosity is not, etc} --- The light of Âtman is
unlike any other light. Ordinary lights are opposed to darkness and are limited
in their capacity to illumine things. It is a common experience that where
there is darkness there is no light; and darkness always prevails at some place
or other, thus limiting the power of illumination of such lights. Even the
light of the sun is unable to dispel darkness at some places. But the light of
Âtman is ever present at all places. It illumines everything and is opposed to
nothing, not even to darkness; for it is in and through the light of Âtman,
which is present in everybody as consciousness, that one comprehends darkness
as well as light and all other things.}

\begin{oframed}

    We are the light that illumines the objects around us and gives us
    experiences. All of the experiences we have all are illumined with the
    light of consciousness. It is the same light that illumines our dreams and
    our waking state. It is the same light that also illumines our deep sleep.
    We experience the absense of everything - that experience is itself because
    of this light. It's not a material light like the light from fire, sun,
    etc. Any sort of mateial light, removes darkness. Consciousness light
    reveals the material light and also the darkness. You know it's dark even
    if there is no material light and that's because of the light of
    consciousness. By its light, everything is lit up. And who is that light?
    Its you!

\end{oframed}


\begin{large}
\begin{center}
    \begin{hindi}

    देहोऽहमित्ययं मूढो धृत्वा तिष्ठत्यहो जनः ।\\
    ममायमित्यपि ज्ञात्वा घटद्रष्टेव सर्वदा ॥ २३ ॥

    \end{hindi}
\end{center}
\end{large}

\texthindi{अहो}
Alas
\texthindi{मूढः}
ignorant
\texthindi{जनः}
person
\texthindi{घटद्रष्टेव}
like a person seeing a pot
\texthindi{ममायमिति}
that this is mine
\texthindi{सर्वदा}
ever
\texthindi{ज्ञात्वा}
knowing
\texthindi{अपि}
even
\texthindi{अहं}
I
\texthindi{अयं}
this
\texthindi{देहः}
body
\texthindi{इति}
that
\texthindi{धृत्वा}
holding (the view)
\texthindi{तिष्ठति}
rests (contented).

\bigskip

\textbf{23. How strange is it that a person ignorantly rests contented with the
idea that he is the body,$^1$ while he knows it as something belonging to him
(and therefore apart from him) even as a person who sees a pot (knows it as
apart from him)!  }

{\small \textit{$^1$The idea that he is the body} --- This is the view of
Laukâyatikas (Indian materialists) who maintain that men is no more than a
fortuitous concourse of material elements. According to them the five elements
of matter, through permutations and combinations, have given birth to this body
as well as to life and consciousness, and with death everything will dissolve
into matter again.

\begin{oframed}

    It's so absured to say, ``the body is mine and I am the body".  If we think
    about it we don't use such language for any other thing - do we say, ``This
    is my bottle and I am the bottle" or ``This is my car and I am the car".
    See how absured it sounds? Then why do we say this about our body? We
    sometimes say it's ``my body" and sometimes we refer it with ourselves
    saying things like, ``I weigh 60 Kgs". So how can something that belongs to
    me, be me? If we say that ``I am this body" and my goal of life is to take
    care of this body and provide it pleasures then I will be miserable.
    Swamiji says it's a tyranny! You are not the body! You are something
    greater than that. We start serving our bodies and forget our true nature
    like ignorant fools - we start pleasing the body. The fool starts holding
    on to this body.

\end{oframed}

\begin{large}
\begin{center}
    \begin{hindi}
    ब्रह्मैवाहं समः शान्तः सच्चिदानंदलक्षणः ।\\
    नाहं देहो ह्यसद्रूपो ज्ञानमित्युच्यते बुधैः ॥ २४ ॥
    \end{hindi}
\end{center}
\end{large}

\texthindi{अहं}
I
\texthindi{ब्रह्म}
Brahman
\texthindi{एव}
verily
(\texthindi{अस्मि}
am
\texthindi{यतः}
because
\texthindi{अहं}
I)
\texthindi{समः}
equanimous
\texthindi{शान्तः}
quiescent
\texthindi{सच्चिदानंदलक्षणः}
by nature absolute Existence, Knowledge and Bliss
(\texthindi{अस्मि}
am)
\texthindi{अहं}
I
\texthindi{असद्रूपः}
non-existence itself
\texthindi{देहः}
the body
\texthindi{नहि}
never
(\texthindi{अस्मि}
am)
\texthindi{इति}
this
\texthindi{बुधैः}
by the wise
\texthindi{ज्ञानम्}
(true) Knowledge
\texthindi{उच्यते}
is called.

\bigskip

\textbf{24. I am verily Brahman,$^1$ being equanimous, quiescent and by nature
absolute Existence, Knowledge and Bliss. I am not the body$^2$ which is
non-existence itself. This is called true Knowledge by the wise.}

{\small \textit{$^1$I am verily Brahman} --- `I,' the Self or Âtman, is
Brahman, as there is not even a single characteristic differentiating the two.
In other words, there are no two entities as Âtman and Brahman; it is the same
entity Âtman that is sometimes called Brahman.

When a person makes an enquiry into the real nature of this external world he
is led to one ultimate reality which he calls Brahman. But an enquiry into the
nature of the enquirer himself reveals the fact that there is nothing but the
Âtman, the Self, wherefrom the so-called external world has emanated. Thus he
realizes that what he so long called Brahman, the substratum of the universe,
is but his own Self, it is he himself. So it is said: `All this is verily
Brahman, this Âtman is Brahman' (\textit{Mând. Up. 2}).

$^2$\textit{I am not the body } --- I am neither the gross, subtle nor the
causal body.
}

\begin{large}
\begin{center}
    \begin{hindi}

    निर्विकारो निराकारो निरवद्योऽहमव्ययः ।\\
    नाहं देहो ह्यसद्रूपो ज्ञानमित्युच्यते बुधैः ॥ २५ ॥

    \end{hindi}
\end{center}
\end{large}

\texthindi{अहं}
I
\texthindi{निर्विकारः}
without any change
\texthindi{निराकारः}
without any form
\texthindi{निरवद्यः}
free from all blemishes
\texthindi{अव्ययः}
undecaying
(\texthindi{अस्मि}
am)
\texthindi{अहम्}
, etc.

\bigskip

\textbf{25. I am without any change, without any form, free from all blemish
and decay. I am not, etc.  }


\begin{large}
\begin{center}
    \begin{hindi}

    निरामयो निराभासो निर्विकल्पोऽहमाततः ।\\
    नाहं देहो ह्यसद्रूपो ज्ञानमित्युच्यते बुधैः ॥ २६ ॥

    \end{hindi}
\end{center}
\end{large}

\texthindi{अहं}
I
\texthindi{निरामयः}
not subject to any disease
\texthindi{निराभासः}
beyond all comprehension
\texthindi{निर्विकल्पः}
free from all alteration
\texthindi{आततः}
all-pervading
(\texthindi{ स्मि}
am)
\texthindi{अहम्}
,etc.

\bigskip

\textbf{26. I am not subject to any disease, I am beyond all comprehension,$^1$
free from all alternatives and all-pervading. I am not, etc.}

{\small \textit{$^1$I am beyond all comprehension} --- I am not comprehended by
any thought, for in the supreme Âtman no thought, the thought of the subject
and the object, the knower and the known, not even the thought of the Self and
the not-Self, is possible, as all thought implies duality whereas the Âtman is
non-dual.}

\begin{oframed}

    \texthindi{निराभासः} is an important and meaning-packed word.  The mind has a
    special quality where it can reflect consciousness. Whatever whe think of
    in the mind or whatever appears in the mind is illumined by the reflected
    consciousness. The analogy here is - the Pure Consciousness is like the sun
    and the mind is like a polished mirror and we can direct the sunlight using
    the mirror whereever we like. That's the quality of the mind. It can
    redirect the light of the consciousness wherever it likes. The
    consciousness we all experience, the feeling of being aware is the
    reflected consiousness.

    Whatever we hear, see, taste, etc creates waves in our minds called
    \texthindi{वृित्ति}
    . To know something we need the reflected consiousness and
    \texthindi{वृित्ति}
    . With the absence of either, we cannot have knowledge about anything. When
    we see a table, the image of a table is created in our minds, that is
    \texthindi{वृित्ति}
    and we come to know about the table because of our reflected consciousness
    - now we have the knowledge of the table.
    Arrising of the
    \texthindi{वृित्ति}
    is called
    \texthindi{विृत्तिव्याप्ति}
    . Illumination of the
    \texthindi{वृित्ति}
    by the reflected consciousness is called
    \texthindi{फलव्याप्ति}
    . Continuing with our analog of the mirror and the sun - we know that to
    know something we focus the beam of sunlight onto the object we would like
    to know about. Now what if we want to know the sun? Do we point the
    reflected beam onto the sun? That sounds so foolish. That's the problem
    with knowing Brahman as well. Brahman is the source of consciousness, tiny
    bit of which we reflect onto this world with our little minds.
    So what do we do to know 'the sun'? -  we just look at it. Just like a
    little reflected beam pointed back at the sun in order to see the sun
    doesn't make any sense, in the same way our minds do not require focus
    upon Brahman in order to know Brahman. This just means that to know
    Brahman is no ordinary Knowledge. To sum up - we do need
    \texthindi{वृित्ति}
    but not
    \texthindi{फलव्याप्ति}
    to know Brahman. The mind has to think about god but we do not require that
    reflected consciousness. This is what is meant by
    \texthindi{निराभासः}
    - without reflection.
    To have the
    \texthindi{वृित्ति}
    of Brahman we need to focus on it - look within, using spiritual practices.

\end{oframed}

\end{document}
